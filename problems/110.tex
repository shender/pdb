\task{Над обрывом установлено орудие, позволяющее вести огонь в любом
  направлении. Снаряды имеют начальную скорость $v$. На расстоянии $l$
  от орудия под углом $\varphi$ к горизонту завис воздушный шар. Известно, что
  шар находится достаточно далеко от орудия --- так, что снаряды в него
  не попадают. Обстрел стали производить снарядами, которые взрываются
  через время $T$ после выстрела. Под каким углом к горизонту следует
  стрелять, чтобы снаряды взрывались как можно ближе к шару? Ускорение
  свободного падения $g$.}
