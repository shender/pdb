\task{ В сосуде объёмом $V_1=20$ л находятся вода, насыщенный водяной
пар и воздух. Объём сосуда при постоянной температуре медленно
увеличивают до $V_2=40$ л, давление в сосуде при этом уменьшается от
$p_1=3$ атм до $p_2=2$ атм. Определите массу воды в сосуде в конце
опыта, если общая масса воды и пара составляет $m=36$ г. Объёмом,
занимаемым жидкостью, в обоих случаях пренебречь. }

% Квант, 2002, №2