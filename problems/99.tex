\taskpic{Параллельные проводящие неподвижные шины расположены в
  горизонтальной плоскости на расстоянии $l$ друг от друга. Однородное
  магнитное поле $B$ направлено вертикально. К шинам подсоединена
  катушка индуктивностью $L$. По шинам может скользить без трения
  проводящая перемычка массой $m$, оставаясь перпендикулярной шинам и
  не теряя с ними электрического контакта. В некоторый момент
  перемычке сообщают скорость $v_0$ вдоль шин. Опишите характер
  движения перемычки.}{
  \begin{circuitikz}
    \draw (0,0) to[L,l^=$L$] (0,2);
    \draw[thick] (0,2) -- (2,2);
    \draw[thick] (0,0) -- (2,0);
    \draw[very thick] (1.5,-0.3) -- (1.5,2.3) node[above,blue] {$m$};
    \draw[->,blue] (1.55,1) -- ++(0.6,0) node[midway,above] {$v_0$};
    \draw[<->,blue] (2.4,0) -- (2.4,2) node[midway,right] {$l$};
    \draw[blue] (0.75,1.75) circle (0.15cm) node[below=0.1cm] {$B$};
    \draw[blue] (0.7,1.8) -- (0.8,1.7);
    \draw[blue] (0.7,1.7) -- (0.8,1.8);
  \end{circuitikz}
}
