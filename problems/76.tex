\task{ В половине шара радиусом $R$ из прозрачного стекла с
  коэффициентом преломления $n=2$ сделано симметричное сферическое
  углубление так, что толщина стекла на линии центров сфер составляет
  $R/2$. Точечный источник света помещён в центре внешней сферической
  поверхности. Где его видит наблюдатель, глаз которого находится
  вдали на линии центров сферических поверхностей?  }