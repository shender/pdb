\taskpic{ На горизонтальной поверхности стола покоится клин массой $M$
  с углом наклона $\alpha$ к горизонту. Со скоростью $v_0$ на него
  наезжает маленькая тележка массой $m$. Через какое время тележка
  съедет с клина? Какое расстояние клин проедет за это время? Въезд на
  клин сделан так, что тележка движется плавно, без толчка. }
{
  \begin{tikzpicture}
    \draw[very thick, interface] (4,0) -- (0,0);
    \draw[thick,fill=gray!20] (2.5,0) -- (3.8,0) -- (3.8,1)
    node[above left,blue] {$M$} -- cycle;
    \draw[thick] (0.5,0.15) circle (0.15cm);
    \draw[thick] (1.3,0.15) circle (0.15cm);
    \draw[fill=green!60] (0.2,0.3) rectangle (1.6,0.5)
    node[above,midway,blue] {$m$};
    \draw[blue,->,thick] (1.7,0.4) -- (2.3,0.4) node[above,blue]
    {$v_0$}; 
    \draw[thick,->] (3,-1) node[left,blue] {$\alpha$} to[out=45,in=15] (2.7,0.07);
  \end{tikzpicture}
}
% Квант, Ф1338, 1992-07
