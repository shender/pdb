\taskpic{
    Пузырек радиусом $r$ поднимается в жидкости и пересекает лазерный луч (луч лежит в вертикальной
    плоскости). На расстоянии $L \gg r$ за пузырьком находится экран. Определите зависимость световой
    точки на экране от времени. Показатель преломления жидкости $n$,скорость пузырька $v$.
}{
    \begin{tikzpicture}
        \draw[blue!20, fill = blue!20] (0, 0) rectangle (3, 2);
        \draw[fill = white] (1.5, 1) circle (0.2cm);
        \draw[thick, <->] (1.5, 0.5) -- (3, 0.5) node[midway, below] {$L$};
        \draw[very thick, marrow, red] (0, 1.3) -- (1.3, 1.3) node[midway, above] {\textit{луч}};
        \draw[platform] (3, 2) -- (3, 0);
    \end{tikzpicture}
}
% Город-2002, 11 класс