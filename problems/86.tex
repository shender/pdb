\task{ В теплоизолированный цилиндрический сосуд поместили кусок льда
  массы $M$ при температуре $t_0=0^{\circ}$C и прочно прикрепили его
  ко дну. Затем залили этот лёд водой такой же массы $M$. Вода
  полностью покрыла лёд и достигла уровня $H=20$~см. Определите,
  какова была температура этой воды, если после установления
  равновесия уровень ее опустился на $b=0.4$~см. Плотности воды и льда
  равны соответственно $\rho_{\text{вода}} =1000 \unit{кг/м}^3$,
  $\rho_{\text{лёд}} = 920 \unit{кг/м}^3$. Удельная теплоёмкость
  $c=4200 \unit{Дж/кг} \cdot C$, удельная теплота плавления льда
  $\lambda = 330$~кДж/кг.  }
