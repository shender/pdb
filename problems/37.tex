\taskpic{Стеклянная пластинка имеет в сечении форму
  равносторонней трапеции. Большее основание трапеции равно $D$,
  высота~---~$L$, угол между боковыми сторонами равен $\varphi \ll
  1$. Боковые поверхности пластинки посеребрены, показатель
  преломления материала пластинки~---~$n$. При каких углах падения
  $\alpha$ луч света, падающий на основание, будет проходить через
  пластинку?}{
  \begin{tikzpicture}
    \draw (0,2) -- (4,2);
    \draw[thick,interface] (1,3.5) -- (3,2.5);
    \draw[thick] (3,2.5) -- (3,1.5);
    \draw[thick,interface] (3,1.5) -- (1,0.5);
    \draw[thick] (1,0.5) -- (1,3.5);
    \draw[thick,dashed] (3,2.5) -- (4,2) -- (3,1.5);
    \draw[thick,postaction=decorate,decoration={markings,mark=at
      position .55 with {\arrow{>}}}] (0,0) -- (1,2);
    \draw[blue] (3.7,2.15) arc (90+atan(2):180+atan(1/2):0.3);
    \draw[blue,->] (3.5,2.75) node[right] {$\varphi$} to[out=180,in=155]
    (3.6,2.1);
    \draw[blue] (0.5,2) arc (180:180+atan(2):0.5);
    \draw[blue] (0.4,1.6) node {$\alpha$};
    \draw[fill=gray!20] (1,0.5) -- (1,3.5) -- (3,2.5) -- (3,1.5) -- (1,0.5);
  \end{tikzpicture}
}