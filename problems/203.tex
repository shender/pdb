\taskpic{ На горизонтальной поверхности стола находится большой и
  очень лёгкий куб. На его верхней гладкой грани помещён кубик массы
  $M$, к которому при помощи лёгких нерастяжимых нитей привязаны
  кубики $M$ и $2M$. Нити переброшены через блоки, закреплённые на
  противоположных рёбрах верхней грани, свисающие концы нитей
  вертикальны. Систему растормаживают. При каком значении коэффициента
  трения между большим кубом и столом этот куб может оставаться
  неподвижным? }
{
  \begin{tikzpicture}
    \draw[interface,thick] (4,0) -- (0,0);
    \draw[fill=black] (0.7,2) circle (0.1) (3.3,2) circle (0.1);
    \draw[fill=white,thick] (0.7,0) rectangle (3.3,2);
    \draw[thick] (1,2) arc (0:270:0.3) (3,2) arc (180:-90:0.3);
    \draw[fill=gray] (1.8,2) rectangle (2.2,2.5)
    node[midway,above=0.3cm,blue] {$M$};
    \draw[thick] (0.7,2.3) -- (1.8,2.3) (2.2,2.3) -- (3.3,2.3);
    \draw[thick] (0.4,2) -- (0.4,1.2) (3.6,2) -- (3.6,1.2);
    \draw[fill=gray] (0.3,1.2) rectangle (0.5,0.7)
    node[midway,below=0.3cm,blue] {$M$};
    \draw[fill=gray] (3.4,1.2) rectangle (3.8,0.7)
    node[below=0.07cm,blue] {$2M$};
  \end{tikzpicture}
}
% Сорос, 2000
