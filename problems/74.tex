\task{ Магнитное поле над постоянным магнитом изменяется с высотой по
  закону $B(h) = B_0 (1- \alpha h)$ (на оси магнита). До какой угловой
  скорости нужно раскрутить в горизонтальной плоскости жёсткое
  непроводящее кольцо, по которому равномерно распределен заряд $q$,
  чтобы оно могло <<парить>> над магнитом? Известно, что масса кольца
  $m$, а его радиус $r$.  }