\taskpic[3cm]{ Два трамвая, двигавшихся навстречу, проходят прямолинейный
  участок путей длиной 480 м со скоростями 10 м/с. Кондукторы ходят от
  задних площадок к передним и обратно со скоростями 2 м/с. Расстояния
  между площадками в вагонах 12 м. В середине участка кондукторы
  поравнялись, находясь на задних площадках. Постройте графики
  зависимостей пути и скорости кондукторов относительно земли от
  времени от момента встречи кондукторов до момента достижения ими
  концов участка. Через сколько времени кондукторы окажутся на концах
  участка, если за начальный момент принять момент встречи
  кондукторов?  }
{
  \begin{tikzpicture}
    \draw[thick] (0,1.5) -- (0,-1.5);
    \draw[thick] (0.2,1.5) -- (0.2,-1.5); 
    \draw[thick] (2,1.5) -- (2,-1.5);
    \draw[thick] (2.2,1.5) -- (2.2,-1.5);
    \draw[thick,fill=white] (-0.1,0.3) rectangle (0.3,-0.8);
    \draw[*->,thick] (0.1,0.2) -- ++(0,-0.8);
    \draw[thick,fill=white] (1.9,-0.1) rectangle ++(0.4,1.1);
    \draw[*->,thick] (2.1,0) -- ++(0,0.8);
  \end{tikzpicture}
}
% Район-2003, 8 класс
% ccpe-2016-2017-8