\taskpic{На пустую катушку магнитофона, делающую одинаковое количество
  оборотов в единицу времени, перематывается магнитная лента. После
  перемотки конечный радиус $r_{\mbox{\textit{к}}}$ намотки оказался в
  семь раз больше начального радиуса $r_{\mbox{\textit{н}}}$. Время
  перемотки ленты равно $t_1$. За какое время $t_2$ на катушку
  перемотается лента такой же длины, но вдвое более тонкая?}{
\begin{tikzpicture}
  \draw[very thick] (2,2) circle (1.25);
  \draw[very thick] (2,2) circle (0.5);
  \draw[blue,dashed] (2,2) -- ++ (0,-1.5);
  \draw[blue,dashed] (3.25,2) -- ++ (0,-1.5);
  \draw[blue,dashed] (2.5,2) -- ++ (0,-0.75);
  \draw[blue,<->] (2,0.5) -- ++(1.25,0) node [midway,below]
  {$r_\text{к}$};
  \draw[blue,<->] (2,1.25) -- ++(0.5,0) node [midway,below] {$r_\text{н}$};
\end{tikzpicture}
}