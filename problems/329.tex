\task{
	Три скрепленных вагончика начинают тянуть по горизонтальным рельсам с постоянной силой, приложенной к первому
    вагончику. Через время $t$ рвется веревка номер 1 (см. рис.) Еще через время $t$ рвется веревка номер 2, но на оставшийся
    вагончик сила действует еще в течение времени $t$. Определите, какие доли от всей работы, совершенной силой, пошли на
    разгон каждого из трех вагончиков. Массы вагончиков одинаковы, изначально они покоятся, трением пренебречь.
}

\begin{center}
    \begin{tikzpicture}
        \draw[thick, interface] (3.8, 0) -- (0, 0);
        \draw[thick] (3.5, 0.1) circle (0.1cm);
        \draw[thick] (3, 0.1) circle (0.1cm);
        \draw[thick] (2.8, 0.2) rectangle ++(0.9, 0.4);
        \begin{scope}[xshift = -1.3cm]
            \draw[thick] (3.5, 0.1) circle (0.1cm);
            \draw[thick] (3, 0.1) circle (0.1cm);
            \draw[thick] (2.8, 0.2) rectangle ++(0.9, 0.4);
        \end{scope}
        \begin{scope}[xshift = -2.6cm]
            \draw[thick] (3.5, 0.1) circle (0.1cm);
            \draw[thick] (3, 0.1) circle (0.1cm);
            \draw[thick] (2.8, 0.2) rectangle ++(0.9, 0.4);
            \draw[thick, ->] (2.8, 0.4) -- ++(-0.5, 0);
        \end{scope}
        \draw[thick] (2.8, 0.4) -- ++(-0.4, 0) node[midway, above] {\small $1$};
        \draw[thick] (1.5, 0.4) -- ++(-0.4, 0) node[midway, above] {\small $2$};
    \end{tikzpicture}
\end{center}
% spb-city-10-2013