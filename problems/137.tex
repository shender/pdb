\task{ Два космических корабля летят с выключенными двигателями в поле
  тяготения звезды, масса которой много больше их масс. Скорости
  кораблей на большом удалении от звезды были равны $v_1$ и $v_2$
  соответственно. После пролёта кораблей около звезды и их удаления на
  большое расстояние от неё векторы их скоростей изменили своё
  направление на $90^{\circ}$ и остались такими же по величине. Первый
  корабль пролетел от звезды на минимальном расстоянии $l_1$. На каком
  минимальном расстоянии от звезды $l_2$ пролетел второй корабль?}
% Московская городская олимпиада, 2006