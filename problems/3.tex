\taskpic{
    Для покоящейся системы, изображенной на рисунке, найдите ускорения всех грузов сразу после того, как была перерезана
    удерживающая их нижняя нить. Считать, что нити невесомы и нерастяжимы, пружины невесомы, масса блока пренебрежимо мала,
    трение в подвесе отсутствует.
}{
    \begin{tikzpicture}
        \draw[interface, thick] (1, 3.5) -- ++(2, 0);
        \draw (2, 3.5) -- ++(0, -1);
        \draw[very thick] (2, 2.5) circle (0.5);
        \draw[thick] (1.5, 2.5) -- ++(0, -0.75);
        \draw[thick] (2.5, 2.5) -- ++(0, -0.75);
        \draw[thick] (1.25, 1.75) rectangle +(0.5, -0.25) node[left = 4pt, midway] {$m_1$};
        \draw[thick] (2.25, 1.75) rectangle +(0.5, -0.25) node[right = 5pt, midway] {$m_3$};
        \draw[spring] (1.5, 1.5) -- ++(0, -0.5) node [midway, left] {$k$};
        \draw[spring] (2.5, 1.5) -- ++(0, -0.5) node [midway, right] {$k$};
        \draw[thick] (1.25, 1) rectangle +(0.5, -0.25) node[left = 4pt, midway] {$m_2$};
        \draw[thick] (2.25, 1) rectangle +(0.5, -0.25) node[right = 5pt, midway] {$m_4$};
        \draw (2.5, 0.75) -- +(0, -0.5);
        \draw[interface, thick] (3, 0.25) -- ++(-0.9, 0);
    \end{tikzpicture}
}
% ММО 1968-1985, 1.40