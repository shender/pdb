\task{На горизонтальной шероховатой поверхности лежит цепочка из $N$
  шариков массы $m$ каждый, связанных пружинками с жесткостью $k$ и
  начальной длиной $l_0 = 0$. Цепочку растягивают и аккуратно
  отпускают. Найдите максимальную возможную длину цепочки, при которой
  все шарики неподвижны. Максимально возможная сила трения между
  поверхностью и шариками равна $F$. Размерами шариков
  пренебречь. Ускорение свободного падения равно $g$.}
% Город-2003, 9 класс