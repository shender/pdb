\taskpic{ По гладкой горизонтальной поверхности скользит маленькая
  круглая шайба, не покидая правильного треугольника, ограниченного
  неподвижными гладкими стенками. Удары шайбы о стенки абсолютно
  упругие, при попадании в угол шайба останавливается. В начальный
  момент шайба находится в точке $A$ посередине стороны треугольника и
  имеет скорость, направленную под углом $\alpha$ к этой стороне, $0<
  \alpha < \pi/2$. Найдите все значения $\alpha$, при которых шайба
  попадёт в угол $B$, совершив при этом не более 6 столкновений со
  стенками. }
{
  \begin{tikzpicture}
    \draw[thick,interface] (0,0) -- ++(60:3cm) node[above,blue] {$B$}
    -- ++(-60:3cm) -- cycle;
    \draw[fill=black,thick] (1.5,0) circle (0.05cm) node[below,blue] {$A$};
    \draw[very thick,->] (1.5,0) -- ++(52:1cm) node[above left,blue]
    {$\vec{v}$};
    \draw[blue] (2,0) arc (0:52:0.5cm) node[right] {$\alpha$};
  \end{tikzpicture}
}
% Туймада, 1.7