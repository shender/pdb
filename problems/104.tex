\taskpic{Три одинаковых источника тепла расположены в цилиндре,
боковые стенки и один из торцов которого теплоизолированы. Второй
торец цилиндра закрыт теплопроводящей мембраной. При наружной
температуре $t_0 = 10^\circ$C в цилиндре устанавливается температура
$t = 25^\circ$C. В цилиндр помещают еще две такие же мембраны,
отделяющие источники друг от друга. Какие температуры установятся в
образовавшихся секциях? Считайте, что мощность теплопередачи
пропорциональна разности температур. Температуру воздуха в пределах
каждой отдельной секции (а до установки дополнительных мембран --- во
всем цилиндре) считайте
одинаковой.}{
\begin{tikzpicture}
  \draw[interface,thick] (3,0) -- (0,0) -- (0,2) -- (3,2);
  \draw (3,0) -- (3,2);
  \draw[dashed] (1,0) -- (1,2);
  \draw[dashed] (2,0) -- (2,2);
  % свечи
  \draw[thick] (0.4,0) rectangle ++(0.2,1);
  \draw[thick] (1.4,0) rectangle ++(0.2,1);
  \draw[thick] (2.4,0) rectangle ++(0.2,1);
  % пламя
  \draw[fill=yellow] (0.5,1) to[out=30,in=-80] (0.5,1.3) to
  [out=-100,in=150] (0.5,1);
  \draw[fill=yellow] (1.5,1) to[out=30,in=-80] (1.5,1.3) to
  [out=-100,in=150] (1.5,1);
  \draw[fill=yellow] (2.5,1) to[out=30,in=-80] (2.5,1.3) to
  [out=-100,in=150] (2.5,1);
\end{tikzpicture}
}
