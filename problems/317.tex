\task{Плосковыпуклая линза сделана из стекла с коэффициентом
преломления $n=1.6$. Радиус сферической поверхности $R=10$ см, толщина
линзы $d=0.2$ см. На плоскую поверхность параллельно главной
оптической оси направляют параллельный пучок и фокусируют его на
экране, открыв только небольшую часть линзы около оси
(<<задиафрагмировав>> линзу). После этого диафрагму убрали. Найти
диаметр пятна на экране.}

% Квант, 1990-10, Ф1237