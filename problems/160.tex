\task{На пробку массой $m_{\mbox{\textit{пр}}}$ намотана проволока из
  алюминия. Плотность пробки равна $\rho_{\mbox{\textit{пр}}} =
  0{,}5\cdot10^3\mbox{ кг/м}^3$, алюминия $\rho_{\mbox{\textit{ал}}} =
  2{,}7\cdot10^3\mbox{ кг/м}^3$, воды $\rho_{\mbox{\textit{в}}} =
  1\cdot10^3\mbox{ кг/м}^3$. Определите, какую минимальную массу
  $m_{\mbox{\textit{ал}}}$ проволоки надо намотать на пробку, чтобы
  пробка вместе с проволокой полностью погрузилась в воду.}
% Московские физические олимпиады, 1968-1985, 1.104