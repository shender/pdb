\task{Маленький шарик массой $m$ и зарядом $q$, брошенный со скоростью
  $v$ под углом $\alpha = 45^\circ$ к горизонту, пролетев вдоль
  поверхности Земли расстояние $L$, попадает в область пространства, в
  которой, кроме поля силы тяжести, имеется ещё и однородное
  постоянное горизонтальное электрическое поле. Граница этой области
  вертикальна. Через некоторое время после этого шарик падает в точку,
  откуда был произведен бросок. Найти напряжённость электрического
  поля $E$. Ускорение свободного падения равно $g$, влиянием воздуха
  пренебречь. }