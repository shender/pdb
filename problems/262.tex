
\taskpic{ Между шариками с массами $m$ и $M$, связанными нитью,
  вставлена лёгкая пружина жёсткостью $k$, сжатая на некоторую
  величину. Система движется со скоростью $V$ вдоль прямой, проходящей
  через центры шариков. Нить пережигают, и один из шариков
  останавливается. Найдите начальную величину сжатия пружины. }
{
  \begin{tikzpicture}
    \draw[thick] (1,0) circle (0.35cm) node {$m$};
    \draw[thick] (3,0) circle (0.4cm) node {$M$};
    \draw (1.35,0) -- (2.6,0);
    \draw[spring] (1.35,0) -- (2.6,0);
    \draw[blue,thick,->] (1.8,0.5) -- (2.5,0.5) node[midway,above] {$V$};
  \end{tikzpicture}
}
% Квант, 2000, №2