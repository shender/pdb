\taskpic{ Плотность масла измеряют в опыте, схема которого показана на
  рисунке. Сосуд разделён на две части вертикальной перегородкой. В
  одну часть сосуда налита вода, в другую --- масло. В перегородку
  встроен шарнир, который может вращаться без трения. В шарнир
  вставлена однородная сосновая линейка, которая находится в
  равновесии. Длина левой части линейки равна $l_1=40$ см, правой ---
  $l_2=60$ см. Плотность воды равна $\rho_0 = 1000 \kgm$, плотность
  линейки $\rho = 600 \kgm$. Найдите плотность масла. }
{
  \begin{tikzpicture}
    \draw[very thick] (0,2) -- (0,0) -- (3,0) -- (3,2);
    \draw[thick] (0,1.8) -- ++(3,0);
    \draw[very thick] (1.5,0) -- ++(0,2);
    \draw[thick] (0.9,1) rectangle ++(1.4,-0.1);
    \draw[fill=white] (1.5,0.95) circle (0.13cm);
    \draw (1.25,1) node[above,blue] {\small $l_1$};
    \draw (1.9,1) node[above,blue] {\small $l_2$};
    \draw (0.5,0.2) node {\small вода};
    \draw (2.3,0.2) node {\small масло}; 
  \end{tikzpicture}
}
% МФО-2007
% ccpe-2016-2017-8