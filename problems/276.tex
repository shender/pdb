\taskpic[5cm]{ В системе, изображённой на рисунке, цилиндрический груз
  массой $m_1$ движется внутри цилиндрического канала чуть большего
  диаметра, просверленного внутри тела массой $m_2$. Нить, соединяющая
  грузы $m_1$ и $m_3$, невесома и нерастяжима, блоки невесомы, трения
  нет. Чему равно ускорение груза массой $m_1$? Ускорение свободного
  падения равно $g$, участок нити между блоками горизонтален.}
{
  \begin{tikzpicture}
    \draw[thick,interface] (3,0) -- (0,0);
    \draw[thick] (0.5,0) rectangle (2.5,2) node[blue,anchor=south
    east,yshift=-2cm,xshift=0.1cm] {$m_2$}; 
    \draw[thick] (1.8,2) -- (1.8,0);
    \draw[thick] (1.2,2) -- (1.2,0);
    \draw[thick] (1.5,2) -- (1.5,1.3);
    \draw[thick] (1.8,2.3) -- (3.7,2.3);
    \draw[thick,fill=white] (1.8,2) circle (0.3cm);
    \draw[thick,fill=black] (1.8,2) circle (0.05cm);
    \draw[thick] (1.22,1.3) rectangle (1.78,0.8) node[midway,blue]
    {\small{$m_1$}};
    \draw[thick,fill=white] (3.7,2) circle (0.3cm);
    \draw[thick,fill=black] (3.7,2) circle (0.05cm);
    \draw[very thick] (3.7,2) -- (3.7,3);
    \draw[thick,interface] (3.3,3) -- (4.1,3);
    \draw[thick] (4,2) -- (4,1.2);
    \draw[thick] (3.6,1.2) rectangle (4.4,0.7) node[blue,midway] {$m_3$};
  \end{tikzpicture}
}
% Москва, Город-2006, 11 класс