\taskpic{Известно, что давление пара над водным раствором сахара
  меньше, чем над чистой водой, на величину $\delta p =
  0.05p_{\text{нас}}C$, где $p_{\text{нас}}$~---~давление насыщенного
  пара над чистой водой, а $C$~---~весовая концентрация
  раствора. Цилиндрический сосуд, наполненный до высоты $h_0=10$~см
  раствором сахара с концентрацией $C_1= 2 \cdot 10^{-3}$, помещают
  под широкий колпак. На дне налит раствор сахара с концентрацией
  $С_2= 10^{-3}$ (см. рис.: высота уровня раствора много меньше
  $h_0$). Каким будет уровень раствора в сосуде после установления
  равновесия? Температура под колпаком поддерживается постоянной и
  равной $20^{\circ}$C. Пар над поверхностью раствора содержит только
  молекулы воды. Универсальная газовая постоянная равна 8.3~Дж/(моль
  К).}{
  \begin{tikzpicture}
    \draw[very thick] (0.2,0) arc (180:0:1.8cm);
    \draw[thick] (1.6,0) -- (1.6,1.3) -- (2.4,1.3) -- (2.4,0) --
    (1.6,0);
    \def\sugar{(1.6,0) -- (1.6,0.9) -- (2.4,0.9) -- (2.4,0) -- (1.6,0)};
    \pattern[pattern=north east lines] \sugar;
    \draw \sugar;
    \def\solone{(0.2,0.07) -- (0.2,0.15) -- (1.6,0.15) -- (1.6,0.07) --
      cycle};
    \def\soltwo{(2.4,0.07) -- (2.4,0.15) -- (3.8,0.15) -- (3.8,0.07) --
      cycle};
    \pattern[pattern=north east lines] \solone;
    \pattern[pattern=north east lines] \soltwo;
    \draw[blue,<->] (1.4,0.15) -- (1.4,0.9) node[left,midway] {$h_0$};
    \draw[line width=0.15cm] (0,0) -- (4,0);
  \end{tikzpicture}
}
