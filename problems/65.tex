% Квант, 1973-10
\task{Имеются две проволоки квадратного сечения, сделанные из одного и
  того же материала. Сторона сечения одной проволоки 1 мм, а другой 4
  мм. Для того, чтобы расплавить первую проволоку, нужно пропустить
  через неё ток 10 А. Какой ток нужно пропустить через вторую
  проволоку, чтобы она расплавилась? Считать, что количество теплоты,
  уходящее в окружающую среду за 1 секунду, пропорционально разности
  температур проволоки и среды, а также площади поверхности проволоки,
  причём коэффициент пропорциональности для обеих проволок одинаков.}