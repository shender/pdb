\task{ Попугай летит с постоянной скоростью по прямой над поверхностью
земли. Над ним на расстоянии 50 м находится орёл. Под ним, на
расстоянии 100 м --- ястреб. Орёл и ястреб одновременно начинают
преследование попугая: их скорости по модулю постоянны, и в каждый
момент времени направлены точно на жертву. Они настигают попугая также
одновременно. Скорость ястреба в два раза больше скорости попугая. Во
сколько раз скорость орла отличается от скорости попугая?}

% Chases and Escapes: The Mathematics of Pursuit and Envasion, p. 32

% Похоже, тут надо интегрировать (хотя и несложно), чтобы найти
% уравнение траектории, а дальше всё просто. 