\taskpic{ В вертикально расположенном сосуде с сечениями $S_1$ и $S_2$
  ($S_1 = 9S_2$) находятся два невесомых поршня. Пространство между
  поршнями заполнено водой. Концы сосуда открыты в атмосферу. К
  верхнему поршню прикреплена пружина жесткостью $k$, к нижнему
  подвешен груз массой $m$. В начальный момент времени пружина не
  растянута, поршни закреплены, расстояние между поршнями $h_0$. На
  сколько просядет верхний поршень, если оба поршня отпустить? } 
{
  \begin{tikzpicture}
    \draw[interface] (0,4) -- (3,4);
    \draw[spring] (1.5,4) -- (1.5,3) node[midway,right,blue] {$k$};
    \draw[thick] (0.7,3.2) -- (0.7,2.2) -- (1.4,2.2) -- (1.4,1);
    \draw[thick] (2.3,3.2) -- (2.3,2.2) -- (1.6,2.2) -- (1.6,1);
    \draw[fill=gray] (0.7,2.8) node[left,blue] {$S_1$} rectangle (2.3,3);
    \draw[fill=gray] (1.4,1.4) node[left,blue] {$S_2$} rectangle (1.6,1.2);
    \draw (1.5,1.2) -- (1.5,0.7);
    \draw[thick] (1.2,0.7) rectangle (1.8,0.2) node[midway,blue]
    {$m$};
    \draw[dashed,blue] (2.3,2.8) -- (3,2.8);
    \draw[dashed,blue] (1.6,1.4) -- (3,1.4);
    \draw[blue,<->] (2.8,2.8) -- (2.8,1.4) node[midway,right,blue]
    {$h_0$};
    % вода
    \draw[fill=blue!10] (0.7,2.8) -- (0.7,2.2) -- (1.4,2.2) --
    (1.4,1.4) -- (1.6,1.4) -- (1.6,2.2) -- (2.3,2.2) -- (2.3,2.8) -- (0.7,2.8);
  \end{tikzpicture}
}
% Санкт-Петербургская городская олимпиада, городской тур, 2003 год