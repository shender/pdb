\task{ Шар радиуса $R$ плавает в воздухе на небольшой высоте над
поверхностью земли. Во сколько раз увеличится модуль силы давления
воздуха на шар, если его радиус увеличить в два раза и при этом он
останется в воздухе примерно на той же высоте? }

% Белорусские физические олимпиады, 2013

% Ответ: в 8 раз.

% Решение: сила давления = сила Архимеда, т.к. шар плавает. Если
% радиус шара увеличить в 2 раза, то объём увеличится в 8 раз, и,
% следовательно, модуль силы давления также увеличится в 8 раз.