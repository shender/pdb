\taskpic { На тонкий тяжелый обруч массой $M=2$ кг, имеющий неподвижную ось
  вращения, намотана длинная нить, под обручем находится утрамбованный
  песок. К нити прикреплен груз массой $m=3$ кг. Когда нить была полностью
  намотана на обруч, высота груза над поверхностью песка составляла $h=2$
  м. Обруч отпускают, при этом груз падает, а веревка
  разматывается. Определить, на какую максимальную высоту поднимется
  груз после того, как, упав в песок, и, некоторое время простояв на
  нем, вновь увлечется нитью вверх.\\ \textit{Указание:} считать, что
  кинетическая энергия тела и вращающегося вокруг оси обруча
  определяется по формуле $mv^2/2$ , где в случае обруча $v$ ---
  скорость точек обода. }
  {
    \begin{tikzpicture}
      \draw[interface] (0.9,3) -- (2.1,3);
      \draw[thick] (1.5,3) -- (1.5,2);
      \draw[very thick] (1.5,2) circle (0.5cm) node[right=0.5cm,blue] {$M$};
      \draw[thick] (2,2) -- (2,1);
      \draw[fill=gray!30] (1.5,1) rectangle (2.5,0.5) node[above right,blue]
      {$m$};
      \draw[thick] (0,-1) -- (3,-1);
      \draw[thick,blue,<->] (1,-1) -- ++(0,1.5) node[left,midway] {$h$}; 
    \end{tikzpicture}
  }
% Киров
