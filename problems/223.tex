\taskpic { Во время ремонта магазина были установлены новые рамы с
  двумя стеклами для витрин, конструкция которых приведена на рисунке:
  толщина $L$ толстого стекла равна 1 см, а тонкого $l=0,5$ см;
  расстояние между стёклами $l_0=2$ см. Одну раму установили толстым стеклом
  внутрь магазина, а другую --- наружу. Какая температура воздуха
  установится между стеклами в каждой из рам, если температура в
  магазине $+20^{\circ}$C, а на улице $-10^{\circ}$C. Считается, что
  теплопередача пропорциональна разности температур, а температура
  воздуха между стёклами из-за конвекции воздуха всюду одинакова. }
{
  \begin{tikzpicture}
    \draw[fill=gray!30] (0,0) rectangle (1,2);
    \draw[fill=gray!30] (3,0) rectangle (3.5,2);
    \draw (2,1) node {воздух};
    \draw[blue,<->] (0,-0.5) -- ++(1,0) node[midway,below] {$L$};
    \draw[blue,<->] (1,-0.5) -- ++(2,0) node[midway,below] {$l_0$};
    \draw[blue,<->] (3,-0.5) -- ++(0.5,0) node[midway,below] {$l$};
  \end{tikzpicture}
}
% Козел, зона 1997, 9 класс
