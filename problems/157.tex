\taskpic{ Цилиндр массой $m$ и радиуса $r$ опирается на две подставки
  одинаковой высот. Одна подставка неподвижна, а другая выезжает
  из-под цилиндра со скоростью $v$. Определите силу давления $N$
  цилиндра на неподвижную подставку в тот момент, когда расстояние
  между точками опоры равно $AB = r \sqrt{2}$. Считать, что в
  начальный момент подставки располагались очень близко друг к другу;
  трением между цилиндром и подставками пренебречь. }
{
  \begin{tikzpicture}
    \draw[interface,thick] (4,1) -- (3,1) node[below right,blue] {$A$} -- (3,0) -- (0,0);
    \draw[thick] (1,0) rectangle (2,1) node[below left,blue] {$B$};
    \draw[thick] (2.5,1.5) circle (0.7);
    \draw[blue,->] (1.5,0.5) -- (0.25,0.5) node[above] {$v$};
    \draw[blue,->] (2.5,1.5) node[below,blue] {$O$} -- ++(30:0.7) node[sloped,blue,midway,above] {$r$};
  \end{tikzpicture}
}
% Московские физические олимпиады 1968-1985, 1.45