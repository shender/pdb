\task{ Туман состоит из огромного количества мельчайших капелек воды,
  неподвижно висящих в воздухе. Масса капелек в 1 л воздуха составляет
  1 г. Маленькая капля воды начинает падать на землю с высоты 5 м,
  <<впитывая>> встреченные капельки. Считая, что капля сохраняет форму
  шарика, найдите её диаметр перед падением на землю.}
% (в качестве подсказки можно дать указание: формулы для объёма
% цилиндра и площади сферы)
% Квант, 2005, №5 (Ф1960)
