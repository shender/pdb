\taskpic[5.5cm]{ Ко дну сосуда при помощи шарнира прикреплена за конец тонкая
  однородная палочка длиной $L$. В сосуд медленно наливают воду и
  отмечают, какая часть длины палочки $L_{\text{п}}$ оказывается под
  водой. График зависимости $L_{\text{п}}/L$ от высоты $h$ уровня
  жидкости над дном сосуда приведён на рисунке. Определите плотность
  материала палочки. Плотность воды равна $\rho_0$. }
{
  \begin{tikzpicture}
    \draw[thick,->] (0,0) node[left] {$0$} -- (4,0) node[right] {$h$};
    \draw[thick,->] (0,0) -- (0,4) node[right] {$L_{\text{п}}/L$} ;
    \coordinate (A) at (1,1);
    \coordinate (B) at (0,1);
    \coordinate (C) at ($(A)+(45:2.5cm)$);
    \draw[very thick, red] (B) node[left,blue] {$\lambda$} -- (A) --
    (C) -- ++(1,0);
    \draw[dashed,thick] (0,0) -- (A) (C) -- ($(0,0)!(C)!(4,0)$)
    node[below,blue] {$L$};
    \draw[dashed,thick] (C) -- ($(0,0)!(C)!(0,4)$) node[left,blue]
    {$1$};
  \end{tikzpicture}
}
% МФО-2006
% ccpe-2016-2017-8