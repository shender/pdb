\task{ На планете Олимптурия-2 температура воды в море Победителей
  составляет зимой $t_1=20^{\circ}$C, а летом $t_2=100^{\circ}$C. В
  качестве высшей меры поощрения, победителей второго тура олимпиады
  по физике в феврале сажают в водолазный колокол (сосуд без дна) и
  опускают на дно моря Победителей. Многолетние наблюдения показали,
  что летом, в процессе подъема победителей из-под колокола начинают
  выходить пузыри, причем происходит это, когда колокол уже вытянут до
  глубины $H=14$ метров. Чему равно атмосферное давление на
  Олимптурии-2, если известно, что в течение года оно практически не
  меняется? Считать, что дыхание победителя не изменяет состав газовой
  смеси под колоколом, температура моря Победителей равна температуре
  воздуха на планете, плотность воды в море считать всегда равной
  $\rho=1000$ кг/м$^3$. }
% Город-1999, 11 класс