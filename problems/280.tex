\taskpic{ На горизонтальной поверхности покоится однородный тонкий
  обруч массой $M$ и радиусом $R$. Горизонтальный диаметр обруча
  представляет собой лёгкую гладкую трубку, в которую помещён шарик
  массой m, прикреплённый к обручу двумя пружинами жёсткостью k
  каждая. Удерживая обруч неподвижным, шарик отклонили влево на
  расстояние $x$, после чего предоставили систему самой себе. Найдите
  ускорение центра обруча в начальный момент времени. Проскальзывание
  обруча отсутствует. }
{
  \begin{tikzpicture}
    \draw[very thick,interface] (4,0) -- (0,0);
    \draw[very thick] (2,1.5) circle (1.5cm);
    \draw[fill=gray] (1.5,1.5) circle (0.25cm) node[blue,above=0.2cm] {$m$};
    \draw[thick] (0.52,1.75) -- (3.48,1.75);
    \draw[thick] (0.52,1.25) -- (3.48,1.25);
    \draw[spring] (3.5,1.5) -- (1.75,1.5)
    node[blue,midway,below=0.2cm] {$k$}; 
    \draw[spring] (0.5,1.5) -- (1.25,1.5)
    node[blue,midway,below=0.2cm] {$k$};
    \draw[dashed,thick] (2,3) -- (2,0);
    \draw[blue,<->] (2,1) -- (1.5,1) node[midway,below] {$x$};
    \draw[dashed,blue] (1.5,1) -- (1.5,1.25);
  \end{tikzpicture}
}
% Москва, город-1988, 10 класс