\task{ Груз неизвестной массы взвешивают, уравновешивая его гирькой с
  известной массой $M$ на концах тяжёлого прямого коромысла; при этом
  равновесие достигается, когда точка опоры коромысла смещается от его
  середины на $x=1/4$ его длины в сторону гирьки. В отсутствие же
  груза на втором плече коромысло остаётся в равновесии при смещении
  его точки опоры от середины в сторону гирьки на $y=1/4$ его
  длины. Считая коромысло однородным по длине, найдите массу
  взвешиваемого груза $m$.} 
% Московские физические олимпиады, 2001, 1.170