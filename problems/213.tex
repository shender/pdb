\taskpic{ Параллельный пучок падает на боковую поверхность стеклянной
  призмы, сечение которой является правильным шестиугольником
  (см. рис.). Точки $A$ и $B$ на рисунке являются серединами
  соответствующих сторон. Пучок преломляется так, что из призмы
  выходят два отдельных параллельных пучка. Найдите минимальный
  показатель преломления материала призмы, при котором такое возможно.
} {
  \begin{tikzpicture}
    \draw[white,fill=blue!20] (-2.5,-0.6) rectangle (-0.75,0.6);
    \node at (0,0) [draw,fill=white,regular polygon, regular
    polygon sides=6, minimum size=2.5cm,outer sep=0pt] {};
    \draw[thick,marrow] (-2.5,-0.6) -- (-0.9,-0.6) node[right,blue] {$B$};
    \draw[thick,marrow] (-2.5,0.6) -- (-0.9,0.6) node[right,blue] {$A$};
  \end{tikzpicture}
}
% Physics Challengs, TPT, May 2012
