\task{ В двухлитровую пластиковую бутыль через короткий шланг
  накачивается воздух до давления 2 атм. Шланг пережимается, и к нему
  присоединяется герметичный тонкостенный полиэтиленовый пакет большой
  ёмкости (больше 10 литров) без воздуха внутри. Бутыль вместе с
  пакетом кладут на одну чашку весов и уравновешивают гирями, которые
  помещают на другую чашку, а затем зажим ослабляется. Воздух из
  бутыли перетекает в пакет, и равновесие весов нарушается. Груз какой
  массы и на какую чашку весов нужно положить, чтобы равновесие весов
  восстановилось? Плотность воздуха равна $1{,}3\mbox{ кг/м}^3$, ускорение
  свободного падения считать равным $10\mbox{ м/с}^2$.}
