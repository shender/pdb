% Квант, 1985-2
\taskpic{
	Пара одинаковых грузиков $A$ и $B$, связанных невесомой нитью длиной $\ell$, начинает соскальзывать с
    гладкого стола высотой $\ell$, причём в начальный момент $B$ находится на высоте $h = 2 \ell / 3$ от
    пола (см. рисунок). Достигнув пола, грузик $B$ прилипает к нему; грузик $A$ в этот момент слетает со
    стола. На какой высоте над уровнем пола будет грузик $A$, когда нить вновь окажется натянутой?
}{
	\begin{tikzpicture}
    	\draw[interface, thick] (4, 0) -- ++(-3, 0);
        \draw[very thick] (1.5, 0) -- ++(0, 2);
        \draw[very thick] (0.75, 2) -- ++(1.5, 0);
        \filldraw (1.6, 2.08) circle (0.07cm) node [blue, above] {$A$};
        \draw (1.6, 2.07) -- (2.25, 2.02) -- ++(0, -0.7);
        \filldraw (2.25, 1.32) circle (0.07cm) node[blue, right] {$B$};
        \draw[densely dashed] (2.25, 0.08) -- ++(40:1.4cm);
        \draw[densely dashed, fill = gray!15] (2.25, 0.08) circle (0.07cm);
        \draw[densely dashed, fill = gray!15] (2.25, 0.08) ++(40:1.4cm) circle (0.07cm);
        \draw[blue, <->] (2.05, 0) -- ++(0, 1.35) node[left, midway] {$h$};
        \draw[blue, <->] (1.2, 0) -- ++ (0, 2) node[left, midway] {$\ell$};
        \draw[blue, <->] (2.55, 0.1) ++(40:1.4cm) -- ++(0, -0.98) node[right, midway] {?};
    \end{tikzpicture}
}
