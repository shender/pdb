\task{ На расстоянии $r$ от заземлённого металлического шарика радиусом
  формула находится точечный заряд $q$. Во сколько раз увеличится сила,
  действующая на шарик, если к нему поднести второй такой же заряд и
  расположить его на расстоянии $r/2$ от шарика так, что отрезки,
  соединяющие шарик с зарядами, будут взаимно перпендикулярны? }

% Наведённый на шарике заряд Q находится из условия равенства нулю
% потенциала его центра. При поднесении второго заряда наведённый
% заряд увеличится в 3 раза. Если вначале на шарик действовала сила
% величиной F притяжения к заряду q, то теперь на него действуют две
% взаимно перпендикулярные силы: величиной 3F со стороны первого
% заряда и величиной 12F со стороны второго.

