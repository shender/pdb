\task{ Три маленьких шарика расположены вдоль оси координат $X$ в
космосе. Вокруг больше ничего нет, гравитационными силами можно
пренебречь по сравнению с электрическими. Скорости всех шариков в
начальный момент равны 0; координаты $x$, $2x$, $4x$; заряды $q$,
$4q$, $9q$; массы $m$, $3m$, $2m$ соответственно. Какими будут
скорости шариков через очень большое время?}

% Турнир Ломоносова-2013