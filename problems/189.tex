\taskpic{ Колесо телеги, равномерно двигающейся по дороге, насажено на
  ось посредством шарикоподшипника. Отношение радиусов внешней и
  внутренней обоймы шарикоподшипника $R:r = 2:1$. Сколько раз шарик
  подшипника обернулся вокруг своей оси, если колесо телеги сделало
  один полный оборот? Шарик между обоймами движется без скольжения.  }
{
  \begin{tikzpicture}
    \draw[very thick] (2,2) circle (0.75cm);
    \draw[very thick] (2,2) circle (1.5cm);
    \foreach \a in {0,60,120,180,240,300}
    { \draw[pattern = north east lines,rotate around={\a:(2,2)}]
      (2,2.75+0.75/2) circle (0.75/2);
    }
    \draw[blue,->] (2,2) -- ++(30:0.75cm) node[below,midway] {$r$};
    \draw[blue,->] (2,2) -- ++(175:1.5cm) node[left] {$R$};
  \end{tikzpicture}
}
% seeds_that_spin
% Ответ: 2
