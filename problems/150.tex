\task{ В стакане с водой плавает деревянная шайба с цилиндрической
  дыркой. Оси шайбы и дырки параллельны. Площадь дна стакана $S$,
  площадь сечения дырки $S_1$. Дырку осторожно наполняют доверху
  маслом. На какую высоту поднимется шайба, если вначале её
  выступающая из воды часть имела высоту $h$? Плотность масла $\rho$,
  плотность воды $\rho_0$. Известно, что всё масло осталось в дырке. }
% Раз задача, два задача, 1.69
