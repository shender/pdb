\task{ В полой сфере проделано маленькое отверстие, через которое
  внутрь проникает узкий параллельный пучок света. Внутренняя
  поверхность сферы отражает свет во все стороны одинаково (диффузно)
  и не поглощает его. Как будут различаться в этом случае освещённости
  в точке, диаметрально противоположной отверстию, и во всех остальных
  точках сферы?  }