\taskpic[3cm]{
    Пластмассовый кубик со стороной $10\unit{см}$ привязан к невесомой нерастяжимой нити, которая
    намотана на катушку. Разматывая катушку, кубик погружают в бассейн с жидкостью. Плотность жидкости
    зависит от глубины. График этой зависимости представлен на рисунке. В самом начале погружения нижняя
    грань кубика касается жидкости. Постройте график зависимости силы натяжения нити от длины ее
    размотанной части. Плотность пластмассы, из которой сделан куб, равна $\rho = 1350\unit{кг/м}^3$.
}{
    \begin{tikzpicture}
        \fill[blue!20] (0.5, 0) -- (2.5, 0) -- (2.5, 1) decorate[water edge]{ -- (0.5, 1)} -- cycle;
        \draw[thin, water edge] (2.5, 1) -- (0.5, 1);
        \draw[thick] (0.5, 1.5) -- (0.5, 0) -- (2.5, 0) -- (2.5, 1.5);
        \draw[thick] (1, 2.5) circle (0.49);
        \fill (1, 2.5) circle (0.05);
        \draw (1, 2.5) -- (1, 3.3);
        \draw[platform] (0.5, 3.3) -- (1.5, 3.3);
        \draw (1.5, 2.5) -- (1.5, 1.1);
        \draw[fill = white, thick] (1.25, 1.1) rectangle (1.75, 0.6);
        \draw[platform] (3, -0.02) -- ++(-3, 0);
    \end{tikzpicture}
}

\begin{center}
    \begin{tikzpicture}
        \begin{axis}[
            grid = both,
            xmin = 0,
            ymin = 1150,
            ymax = 1450,
            y = 0.015cm,
            x = 0.25cm,
            xlabel = {Глубина, $\unit{см}$},
            ylabel = {Плотность, $\unit{кг/м}^3$},
            ytick = {1200, 1250, ..., 1450}]
            \addplot[mark = none, very thick, blue, line width = 2pt, domain = 0:45] {1200 + 5 * x};
        \end{axis}
    \end{tikzpicture}
\end{center}
% Город-2004, 7 класс
