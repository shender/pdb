\task{В цилиндрический сосуд засыпают деревянные шарики общей массой
  $m = 500$~кг. Затем шарики вынимают, и в сосуд заливают воду массой
  $M = 1000$~кг, причем она достигает того же уровня, что и шарики до
  этого --- $h = 1$~м от дна. Шарики засыпают обратно. На каком
  расстоянии от дна будут находиться самые верхние шарики? Плотность
  дерева $\rho = 800$~кг/м$^3$, плотность воды $\rho_0 =
  1000$~кг/м$^3$.}
% Район-2007, 9 класс