\taskpic{ Грузик подвешен в точке $D$ на трёх одинаковых пружинах,
  закреплённых на горизонтальной линии в точках $A$, $B$ и $C$, причём
  расстояние $AB=BC$ и равно длине недеформированной пружины. В
  положении равновесия углы $ADB = BDC = 30^{\circ}$. Внезапно пружина
  $AD$ разорвалась. Найти модуль и направление ускорения грузика сразу
  после разрыва. Массой пружин пренебречь. }
{
  \begin{tikzpicture}
    \draw[interface] (0,3) -- (4,3);
    \draw[spring] (2,1) --++(60:2.4cm) node[above] {$C$};
    \draw[spring] (2,1) --++(120:2.4cm) node[above] {$A$};
    \draw[spring] (2,1) --++(90:2.07cm) node[above] {$B$};
    \draw[fill=black] (2,0.9) circle (0.2cm) node[right=0.2cm] {$D$};
  \end{tikzpicture}
}
% НГУ, 1.38
