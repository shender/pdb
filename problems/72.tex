\task{ Плоский конденсатор ёмкости $C$ составлен из двух больших
  проводящих пластин, каждая из которых сделана <<двухслойной>> --- из
  соединённых друг с другом листов тонкой фольги. Пластины несут
  одноимённые заряды $Q$ и $2Q$. Наружный слой фольги пластины с
  большим зарядом аккуратно отсоединяют, относят в сторону параллельно
  другим пластинам и переносят на другое место --- <<третьим слоем>>,
  снаружи, к пластине с зарядом $Q$. При этом не допускают
  электрического контакта с этой пластиной --- оставляют очень узкий
  зазор. Какую работу необходимо при этом совершить? Все действия мы
  производим издали, стараясь не влиять на распределение зарядов
  пластин.  }