\task{
  Предположим, что создан материал с необычной зависимостью
  коэффициента теплопроводности $k$ от температуры. Пластину из такого
  материала поместили между двумя стенками вплотную к ним. Температуры
  стенок поддерживаются постоянными: $T_1 = 160 K$ и $T_2 = 500 K$
  соответственно. Какой тепловой поток установится между стенками,
  если толщина пластины $d=1 \mbox{ см}$, а её площадь $S= 100 \mbox{
    см}^2$? Найдите температуру в среднем продольном сечении пластины
  ($x=d/2$). \\
\textit{Указание.} Тепловой поток $P$ сквозь тонкий слой вещества,
площадь которого равна $S$, а толщина $dx$, равный количеству теплоты,
проходящему через этот слой в единицу времени, прямо пропорционален
разности значений температуры его поверхностей $dT$ и обратно
пропорционален его толщине: $P = - k S \dfrac{dT}{dx}$, где $k$ ---
коэффициент теплопроводности вещества. 
}

\begin{figure}[h]
  \centering
  \begin{tikzpicture}[>=latex,scale=1.2]
    \draw[help lines,xstep=0.6,ystep=0.35] (0,0) grid (3.4,3.5);
    \draw[thick,->] (0,0) -- (3.6,0) node[above=5,left=-5] {\tiny{$T$, K}};
    \draw[thick,->] (0,0) -- (0,4) node[right=23,below] {\tiny{$k$,Вт/(м$\cdot$К)}};
    \coordinate (a) at (0.8,3);
    \coordinate (b) at (3.2,2.3);
    \coordinate (c) at (1.5,4.3);
    \coordinate (d) at (2.2,0);
    \draw[very thick,red] (a) .. controls (c) and (d) .. (b);
    \draw[blue,dashed] (0.92,0.35*9) -- ++(0,-0.35*9);
    \draw[blue,dashed] (0.6*5,0.35*5.5) -- ++(0,-0.35*5.5) node [below=-1] {\tiny{$T_2$}};
    \draw[blue,->] (0.5,-0.2) node[left=-3] {\tiny{$T_1$}} to
    [out=0,in=215] (0.9,-0.05);
    \draw[thick] (0.6*2,0.15/2) -- ++(0,-0.15) node[below] {\tiny{$200$}};
    \draw[thick] (0.6*4,0.15/2) -- ++(0,-0.15) node[below]
    {\tiny{$400$}};
    \draw[thick] (0.1,0.35*5) -- ++(-0.2,0) node[left=-3] {\tiny{$1$}}; 
    \draw[thick] (0.1,0.35*10) -- ++(-0.2,0) node[left=-3] {\tiny{$2$}};  
  \end{tikzpicture}
\end{figure}
